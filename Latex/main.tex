\documentclass{meta/masterthesis}
% masterthesis for Computer Science
% masterthesis_ceng for Computer Engineering
% use the [nocoverpage] option to speed up compilation by skipping the cover page

% the configuration is in the preamble.tex file, read it and modify it to change
% the document appearance and included packages
\input{meta/preamble}

% by default, we use the biblatex package to manage the bibliography
% if you prefer to use bibtex, comment the following line and uncomment the
% relevant lines at the end of the document
\addbibresource{meta/bib.bib}

\acrodef{DIBRIS}{\textit{dipartimento di informatica, bioingegneria, robotica e ingegneria dei sistemi}}
\acrodef{LLM}{large language model}

\acrodef{NSA}{National Security Agency}
\acrodef{GUI}{Graphical User Interface}
\acrodef{RTL}{Register Transfer Language}
\acrodef{SSA}{Static Single Assignment}
\acrodef{CFG}{Control Flow Graph}

\begin{document}

\title{Using LLMs to Rank Decompiled Code Variants}

\author{Luigi Timossi}

% important for when you'll submit your thesis: your advisor(s) are "relatore", and the examiner is "correlatore".

\advisor{Matteo Dell'Amico, Giovanni Lagorio}

\examiner{Marina Ribaudo}

\maketitle

\begin{abstract}
Reverse engineering relies heavily on decompilers to translate compiled binaries back into readable high-level code. However, assessing the quality and ``human-likeness'' of decompiled output remains a highly subjective and challenging task. Traditional metrics fall short of capturing readability, idiomatic structures, and semantic clarity. This thesis investigates the use of \acp{LLM} to automatically evaluate and rank decompiled code variants. We compare two main approaches: using intrinsic statistical metrics, specifically perplexity, and employing \acp{LLM} as qualitative judges (LLM-as-a-Judge). We evaluate two open-weight models, \emph{qwen-3} and \emph{deepseek-r1}, on a dataset of real-world C projects decompiled with Ghidra (comparing different commits and Pull Requests) and other decompilers via Dogbolt (Binary Ninja, Hex-Rays).

Our findings reveal that perplexity is a poor proxy for human-likeness, as human-authored code naturally exhibits higher entropy and structural variance compared to the rigid, repetitive boilerplate generated by decompilers. Conversely, the LLM-as-a-Judge approach shows significant promise: \emph{deepseek-r1} achieved an alignment of $\sim$74\% with human developers' judgments. However, we also identify critical vulnerabilities in \acp{LLM}, such as lexical bias, verbosity, and analytical hallucinations when faced with minor stylistic differences. We demonstrate that abstracting the code into an \ac{AST} effectively mitigates some of these lexical biases. Ultimately, this work establishes a robust baseline for automated decompiler evaluation and highlights both the potential and the limitations of using \ac{LLM} in reverse engineering workflows.
\end{abstract}

\tableofcontents

\chapter{Introduction}\label{ch:introduction}

Reverse engineering is a critical process in software security, enabling analysts to understand, debug, and modify software without access to its source code~\cite{Paper:chikofsky1990reverse}.
Decompilation tools like \emph{Ghidra} and \emph{Hex-Rays} have long been the backbone of this process, translating binary executables back into high-level code~\cite{Paper:eagle2011idapro, site:ghidra2019}.
However, the output from these tools often suffers from issues such as poor readability, non-idiomatic constructs, and a lack of meaningful variable names, 
which can significantly hinder the analyst's ability to comprehend and work with the decompiled code.
 
The advent of \ac{LLM} has opened new avenues for enhancing reverse engineering workflows. \ac{LLM} have demonstrated remarkable capabilities in understanding and generating code, making them promising candidates for improving 
the quality of decompiled output. Recent research has explored using \ac{LLM} to refine decompiler output, generate comments, and even act as judges 
to evaluate code quality. However, much of this work has focused on either generative refinement or broad benchmarking of decompilers, 
often relying on proprietary models and tools~\cite{Paper:Tan_2024, Paper:Hu2024DeGPTOD}.
In this thesis, we take a different approach by leveraging \ac{LLM} to evaluate the ``humanness'' of decompiled code using intrinsic model 
metrics like \emph{perplexity}~\cite{Paper:hindle2012naturalness}. We investigate how well local \ac{LLM} can distinguish between different versions of the same codebase, 
such as \emph{pull requests}, without modifying the code itself. This fine-grained analysis is crucial for assessing incremental changes in code 
quality and readability, which is often more relevant in real-world reverse engineering tasks than wholesale comparisons of different decompilers.
Our work also addresses the practical constraints of reverse engineering, such as privacy and cost, by exploring the feasibility of running these 
evaluations on local hardware, rather than relying on cloud-based \ac{API}s. This makes our approach more accessible and applicable in security-sensitive contexts 
where data privacy is paramount~\cite{Paper:staab2024beyond, Paper:carlini2021extracting}.

This thesis explores the application of \ac{LLM} to automate the evaluation and ranking of decompiled code variants. Specifically, we investigate whether \acp{LLM} can effectively proxy human judgment in determining which decompiled variant is more readable, idiomatic, and structurally sound. 
For limitations of time and resources, we focus on two open-weight models, \emph{qwen-3} and \emph{deepseek-r1}, which are representative of the current state-of-the-art in local \ac{LLM} capabilities in the range of 14B of parameters.
We explore two distinct methodologies: using the statistical predictability of the code (Perplexity) as a quantitative metric, and prompting the models to act as qualitative evaluators (LLM-as-a-Judge). 
By testing on real-world datasets across different decompilers and specific pull requests within the Ghidra ecosystem, we aim to uncover the biases, strengths, and limitations of this approach, creating a framework with anonimizations and bias limitations.

\chapter{Related Work}
\label{ch:related}

The intersection of \ac{LLM} and reverse engineering has rapidly evolved, 
transforming how analysts interact with decompiled code. While recent 
literature has indeed explored utilizing LLMs to assist in reverse engineering (using framework like MCP servers), 
the vast majority of this work focuses on two distinct areas: 
\begin{itemize}
    \item Generative refinement of decompiler output
    \item LLM-based evaluation and benchmarking of decompilation tools
\end{itemize}

Unlike generative approaches that aim to produce better code, our work focuses on 
measuring the "humanness" of existing code using intrinsic model metrics like perplexity. 
Furthermore, unlike broad benchmarking frameworks that rely on proprietary APIs to rank tools, 
our research investigates the granular utility of local \ac{LLM} in distinguishing between specific versions (or pull requests) of the same codebase.

\section{Generative Refinement of Decompiler Output}
\label{sec:genref}
The most prominent use of LLMs in this field is the attempt to improve the readability of the raw 
output produced by traditional decompilers (like Ghidra or Hex-Rays). This body of work is complementary 
to ours; while we do not attempt to modify the code, understanding the deficits of raw decompiler output 
explains why our metrics (such as perplexity) are necessary to quantify "humanness."

Some example of this approach are LLM4Decompile\cite{Paper:Tan_2024} wich is an \ac{LLM} model
that was trained to decompile binary code into high-level language, acting as a decompiler itself 
(LLM4Decompile-End is the model to decompile, LLM4Decompile-Ref is the model to refine another decompiler output); 
or DeGPT \cite{Paper:Hu2024DeGPTOD}, which introduces an end-to-end 
framework designed to optimize decompiler output directly employing a "three-role mechanism" 
(Referee, Advisor, and Operator) to guide an LLM in renaming variables, appending comments, 
and simplifying structure. 

Their work demonstrates that LLMs can significantly reduce the 
cognitive burden on analysts by rewriting code to be more idiomatic.

\section{LLM-based Benchmarking}
\label{sec:llmbench}

Closer to our specific problem domain is the emerging field of using \ac{LLM}s to evaluate code quality, 
using \ac{LLM}s to scale and automatize human evaluation, this technique is often referred to as "LLM-as-a-Judge." \cite{Paper:li2025generationjudgmentopportunitieschallenges}

DecompileBench \cite{Paper:gao2025decompilebenchcomprehensivebenchmarkevaluating} is the state-of-the-art in this area,
It presents a comprehensive framework for evaluating decompilers, introducing the concept of using "LLM-as-a-Judge" to rate code 
understandability across 12 specific dimensions (e.g., Variable Naming, Control Flow Clarity). 
Their work validates that LLMs can align well with human experts in ranking different decompilers 
(e.g., comparing Ghidra vs. Hex-Rays vs. LLM-based decompilers).
While DecompileBench is similar to our work in its use of LLMs for assessment, differ from ours on the type of validation used: 
DecompileBench relies only on prompting the model to output a score based on Function Source Code, Decompiler A’s Pseudo Code, and Decompiler B’s Pseudo Code to calculate an ELO rating.
In contrast, our work leverages also with intrinsic model metrics like \texttt{perplexity} to quantify code "humanness". 
Perplexity provides a more objective, quantifiable measure of how "natural" the code appears to the model, 
rather than relying on subjective ratings.[\ref{sec:perplexity}][\ref{sec:WhyPerp}]

Our work also diverges from DecompileBench in its focus on evaluating code variants
within the same codebase (e.g., different pull requests), rather than comparing entirely different decompilers. 
This fine-grained analysis is crucial on cases where its required to assess incremental changes rather than wholesale tool comparisons (es. GitHub Actions \cite{site:githubGitHubActions}).

The last difference is that DecompileBench heavily utilizes proprietary, closed-source models (like GPT-4 
or Claude-3.5) and licensed decompilers (like Hex-Rays or Bininja). 
Our work specifically explores the feasibility of running these evaluations on local hardware. 
This addresses the privacy and cost constraints often present in security-sensitive reverse engineering tasks, 
which large-scale benchmarks often overlook.

\section{Why Perplexity}
\label{sec:WhyPerp}
If we accept that human source code is "natural" and predictable, we can model it stochastically 
using neural (Transformer) language models. From this perspective, the decompiler acts as a noisy 
channel that introduces distortions into the original signal. The goal of LLM-based evaluation is to 
quantify how much the output signal (the decompiled code) deviates from the expected statistical 
distribution of "natural" human code. 

A language model trained on human source code learns a probability distribution $P$ over token sequences. 
When this model observes a sequence of decompiled code $S = t_1, t_2, \ldots, t_N$, it assigns a probability to each token based on the preceding context. 
If the decompiled code uses alien or "unnatural" constructs, the model, expecting human patterns, will assign these tokens a very low probability. 
This statistical "surprise" is the foundation of perplexity.


\chapter{Background}
\label{ch:background}


summary of all the things ghidra + llm

\section{The Ghidra Architecture}
\label{sec:ghidra_arch}

Ghidra, released by the \ac{NSA} in 2019, employs a bifurcated design
that separates the user-facing interaction layer from the core analysis engine. This separation is
not merely an implementation detail but a fundamental architectural constraint that dictates how data 
flows during the reverse engineering process.

The framework operates across two distinct memory spaces: a frontend implemented in Java and a backend
analysis engine written in C++. The Java frontend is responsible for the \ac{GUI},
project database management, and plugin orchestration. It provides the high-level API exposed to users
and scripts (e.g., Python or Java scripts via the GhidraScript framework). However, the computationally
intensive tasks of data-flow analysis, variable inference, and control flow structuring are offloaded to
a native C++ executable, typically named \texttt{decomp} or \texttt{decomp\_dbg} (for debugging).
These executables and the code are located at \texttt{Ghidra/Features/Decompiler/src/decompile/cpp}.


Communication is mediated by the \texttt{ghidra.app.decompiler.DecompInterface}. 
This interface manages a dedicated input/output stream to the native process, utilizing an XML-based protocol to exchange data.
When a function decompilation is requested, the Java client does not simply invoke a library function; it serializes the request into
an XML command (e.g., \texttt{<decompile\_at>}) and transmits it to the backend. The C++ process, holding its own representation
of the function's data flow in \texttt{Funcdata} objects, performs the analysis and returns the results as a serialized XML stream
describing the high-level code structure and syntax tokens.

\section{SLEIGH and P-code}
\label{sec:slap}

As written in the documentation created by running \texttt{<make doc>} \cite{DOC:shareGhidraDecompiler} The 
decompiler provides its own \ac{RTL}, referred to internally as p-code,
which is designed specifically for reverse engineering applications. The disassembly of processor
specific machine-code languages, and subsequent translation into p-code, forms a major sub-system 
of the decompiler. There is a processor specification language, referred to as SLEIGH, which is 
dedicated to this translation task, this piece of the code can be built as a standalone binary 
translation library, for use by other applications.

\subsection{P-code Semantics and Varnodes}
Unlike intermediate languages in compilers, P-code is designed specifically for reverse engineering, prioritizing the explicit representation of memory and register modifications.

The fundamental unit of data in P-code is the \textbf{Varnode}. A Varnode is defined by the triple $(Space, Offset, Size)$, representing a contiguous sequence of bytes in a specific address space.

\begin{table}[ht]
    \centering
    \caption{Some P-code Operations and Semantics \texttt{opcodes.hh}\cite{DOC:shareGhidraDecompiler}\cite{DOC:spinselPCodeReference}}
    \label{tab:pcode_ops}
    \begin{tabular}{l p{0.25\textwidth} p{0.45\textwidth}}
        \toprule
        \textbf{Opcode} & \textbf{Operands} & \textbf{Semantics} \\
        \midrule
        \texttt{CPUI\_COPY } & $in_0 \rightarrow out$ & Copy one operand to another. \\
        \texttt{CPUI\_LOAD } & $space, ptr \rightarrow out$ & Load from a pointer into a specific address. \\
        \texttt{CPUI\_STORE} & $space, ptr, val$ & Store at a pointer into a specified address space. \\
        \texttt{CPUI\_INT\_ADD} & $in_0, in_1 \rightarrow out$ & Integer addition, signed or unsigned. \\
        \texttt{CPUI\_CBRANCH} & $dest, cond$ & Conditional jump to $dest$ if $cond$ is non-zero. \\
        \bottomrule
    \end{tabular}
\end{table}

We must distinguish between two forms of P-code used during analysis:
\begin{enumerate}
    \item \textbf{Raw P-code:} The direct, unoptimized output of the SLEIGH translation. 
    It is represented by the class \textbf{PcodeOpRaw} (or by unprocessed PcodeOp), 
    and contains the bare essentials: an opcode, a sequence number (address), and 
    the input/output Varnodes.
    \item \textbf{High P-code:} The result of the analysis pipeline. In this form, 
    the code has been converted to \ac{SSA} form (a form where every varnode is defined 
    exactly once for each function, if a variable is assigned multiple times, each assignment 
    is given a new instance called low-level variable), dead code has been eliminated, 
    and high-level concepts like function calls (replacing jump-and-link semantics) 
    have been recovered. It is represented by the class \textbf{HighVariable}; 
    this is an abstraction that groups multiple low-level Varnodes (which may reside 
    in different registers or stack locations during execution) into a single logical 
    variable, similar to a variable in C code.
\end{enumerate}

The transformation from Raw to High P-code is where the majority of the decompilation logic resides.
It is an inference process that attempts to raise the abstraction level of the code,
often relying on heuristics that may fail in the presence of obfuscation
or aggressive compiler optimizations.

\section{The Decompilation Pipeline}
\label{sec:pipeline}
The C++ decompiler engine processes a function at a time through a series of iterative passes. 
The architecture organizes these passes into \textbf{Actions} and \textbf{Rules}, 
managed by the \texttt{ActionDatabase}.
inside the \texttt{ActionDatabase::universalAction} we have two main types of objects:
\begin{itemize}
    \item \texttt{ActionGroup}: Represents a list of Actions that are applied sequentially. The group's properties (eg., rule\_repeatapply) influence how the contained actions are executed.
    \item \texttt{ActionPool}: It is a pool of Rules that are applied simultaneously to every PcodeOp. Each Rule triggers on a specific localized data-flow configuration. The Rules are applied repeatedly until no Rule can make any additional transformations.
\end{itemize}

\subsection{Actions and Rules}
Actions represent large-scale transformations applied to the graph of varnodes and operations. They are the base class for objects that make modifications to a function's (Funcdata) syntax tree. Their purpose is to manage complex stages of the workflow, such as recovering the control-flow structure or generating \ac{SSA} form.

Rules, on the other hand, are a class designed to perform a single specific transformation on a PcodeOp or a Varnode. A Rule triggers when it recognizes a particular local configuration in the data flow and specifies a sequence of modification operations to transform it.

\subsection{DefaultGroups}

Actions and Rules are selected and activated according to the type of \textbf{DefaultGroup} they belong to.
These groups represent standardized workflows for different analysis phases and are built by the method \texttt{ActionDatabase::buildDefaultGroups()}. The main groups are:

\begin{itemize}
    \item \textbf{decompile}: the standard workflow for full decompilation, composed of all of the phases.
    \item \textbf{jumptable}: optimized for analyzing jump tables.
    \item \textbf{normalize}: used for code normalization.
    \item \textbf{paramid}: for parameter identification.
    \item \textbf{register}: for register analysis.
    \item \textbf{firstpass}: a first fast analysis pass.
\end{itemize}

Each DefaultGroup is a list of names that refer to specific \texttt{ActionGroup}, \texttt{ActionPool} or individual \texttt{Action} to execute in that configuration. These lists define subsets of all the Actions.

The decompiler can be customized by selecting different DefaultGroups in java with the method \texttt{setSimplificationStyle()} of the decompiler interface but
Only the group named \textbf{decompile} return C code to ghidra, since in \texttt{ghidra\_process.cc} we have:

\begin{lstlisting}[language=C++, caption={ghidra\_process.cc}]
    [...]
      fd->encode(encoder,0,ghidra->getSendSyntaxTree());
      if (ghidra->getSendCCode()&&
	  (ghidra->allacts.getCurrentName() == "decompile"))  //HERE WE HAVE THE CHECK
        ghidra->print->docFunction(fd);
    [...]
\end{lstlisting}

\section{Logic of Control Flow Structuring}
\label{sec:cfg_structuring}

Recovering high-level control structures (loops, conditionals) from the unstructured \ac{CFG}
is arguably the most challenging phase of decompilation. It is effectively a pattern-matching
problem on a directed graph, aimed at finding subgraphs that correspond to structured programming constructs.

\subsection{Basic Block Formulation}
The decompiler first aggregates P-code operations into \textbf{BasicBlocks} sequences of instructions with a single entry point 
and a single exit point (excluding internal calls). The \ac{CFG} is formed by the edges representing jumps and branches between these blocks. 
Ghidra normalizes this graph to ensure a unique entry block, often inserting empty placeholder blocks to handle re-entrant loops or complex function entries.

\begin{figure}
    \centering
    \includegraphics[width=0.4\textwidth]{img/cfg1.png}
    \caption{Control Flow Graph of a function}
    \label{img:cfg1}
\end{figure}

In this example we have the C code of the function described in \ref{img:cfg1} and its corresponding P-code representation \footnote{
    P-codes varies during all phases of the decompilation process; due to optimization rules, dead code elimination, and other transformations, the P-code shown here are taken from the \texttt{collapseInternal()} method using \texttt{printRaw()} of the FlowBlock class.
    The BasicBlock order may not correspond directly to the original source code order
}.

\begin{longtable}{|p{0.30\linewidth}|p{0.65\linewidth}|}
    \hline
    \textbf{Source Code (C)} & \textbf{P-Code / Basic Blocks} \\ 
    \hline
    \endhead % Ripete l'intestazione se cambia pagina

    % --- ROW 1 (Block 0) ---
    \begin{minipage}[t]{\linewidth}
        \vspace{1mm} % Piccolo padding superiore
        \begin{lstlisting}[style=CStyle]
int a2_local;
int a1_local;
putchar(L'1');
if ((a1 == 1)
        \end{lstlisting}
        \vspace{1mm}
    \end{minipage} 
    & 
    \begin{minipage}[t]{\linewidth}
        \vspace{1mm}
        \textbf{\scriptsize Basic Block 0}
        \begin{lstlisting}[style=PCodeStyle]
0x0010118d:1:	RSP(0x0010118d:1) = RSP(i) + #0xfffffffffffffff8
0x0010118d:2:	*(ram,RSP(0x0010118d:1)) = RBP(i)
0x00101195:d:	u0x00004780(0x00101195:d) = RSP(i) + #0xfffffffffffffff4
0x00101195:f:	*(ram,u0x00004780(0x00101195:d)) = EDI(i)
0x00101198:10:	u0x00004780(0x00101198:10) = RSP(i) + #0xfffffffffffffff0
0x00101198:12:	*(ram,u0x00004780(0x00101198:10)) = ESI(i)
0x001011a0:14:	RSP(0x001011a0:14) = RSP(i) + #0xffffffffffffffe0
0x001011a0:15:	*(ram,RSP(0x001011a0:14)) = #0x1011a5
0x001011a0:67:	u0x10000008:1(0x001011a0:67) = *(ram,RSP(0x001011a0:14))
0x001011a0:16:	call jputchar(free)(#0x31:4,u0x10000008:1(0x001011a0:67))
0x001011a5:17:	u0x00004780(0x001011a5:17) = RSP(i) + #0xfffffffffffffff4
0x001011a5:18:	u0x00011e80:4(0x001011a5:18) = *(ram,u0x00004780(0x001011a5:17))
0x001011a5:1e:	ZF(0x001011a5:1e) = u0x00011e80:4(0x001011a5:18) == #0x1:4
0x001011a9:23:	goto Block_2:0x001011bd if (ZF(0x001011a5:1e) != 0) else Block_1:0x001011ab
        \end{lstlisting}
        \vspace{1mm}
    \end{minipage} 
    \\ \hline

    % --- ROW 2 (Block 1) ---
    \begin{minipage}[t]{\linewidth}
        \vspace{1mm}
        \begin{lstlisting}[style=CStyle]
|| (a2 != 2)){
        \end{lstlisting}
        \vspace{1mm}
    \end{minipage}
    & 
    \begin{minipage}[t]{\linewidth}
        \vspace{1mm}
        \textbf{\scriptsize Basic Block 1}
        \begin{lstlisting}[style=PCodeStyle]
0x001011ab:24:	u0x00004780(0x001011ab:24) = RSP(i) + #0xfffffffffffffff0
0x001011ab:25:	u0x00011e80:4(0x001011ab:25) = *(ram,u0x00004780(0x001011ab:24))
0x001011ab:2b:	ZF(0x001011ab:2b) = u0x00011e80:4(0x001011ab:25) != #0x2:4
0x001011af:31:	goto Block_2:0x001011bd if (ZF(0x001011ab:2b) != 0) else Block_4:0x001011b1
        \end{lstlisting}
        \vspace{1mm}
    \end{minipage}
    \\ \hline

    % --- ROW 3 (Block 2) ---
    \begin{minipage}[t]{\linewidth}
        \vspace{1mm}
        \begin{lstlisting}[style=CStyle]
putchar(L'2');
if (a1 != a2) {
        \end{lstlisting}
        \vspace{1mm}
    \end{minipage}
    & 
    \begin{minipage}[t]{\linewidth}
        \vspace{1mm}
        \textbf{\scriptsize Basic Block 2}
        \begin{lstlisting}[style=PCodeStyle]
0x001011c2:46:	RSP(0x001011c2:46) = RSP(i) + #0xffffffffffffffe0
0x001011c2:47:	*(ram,RSP(0x001011c2:46)) = #0x1011c7
0x001011c2:69:	u0x10000011:1(0x001011c2:69) = *(ram,RSP(0x001011c2:46))
0x001011c2:48:	call jputchar(free)(#0x32:4,u0x10000011:1(0x001011c2:69))
0x001011c7:49:	u0x00004780(0x001011c7:49) = RSP(i) + #0xfffffffffffffff4
0x001011c7:4a:	u0x00011e80:4(0x001011c7:4a) = *(ram,u0x00004780(0x001011c7:49))
0x001011ca:4d:	u0x00004780(0x001011ca:4d) = RSP(i) + #0xfffffffffffffff0
0x001011ca:4e:	u0x00006a00:4(0x001011ca:4e) = *(ram,u0x00004780(0x001011ca:4d))
0x001011ca:54:	ZF(0x001011ca:54) = u0x00011e80:4(0x001011c7:4a) == u0x00006a00:4(0x001011ca:4e)
0x001011cd:59:	goto Block_3:0x001011cf if (ZF(0x001011ca:54) == 0) else Block_5:0x001011db
        \end{lstlisting}
        \vspace{1mm}
    \end{minipage}
    \\ \hline

    % --- ROW 4 (Block 3) ---
    \begin{minipage}[t]{\linewidth}
        \vspace{1mm}
        \begin{lstlisting}[style=CStyle]
putchar(L'4');
goto LAB_001011e5;
        \end{lstlisting}
        \vspace{1mm}
    \end{minipage}
    & 
    \begin{minipage}[t]{\linewidth}
        \vspace{1mm}
        \textbf{\scriptsize Basic Block 3}
        \begin{lstlisting}[style=PCodeStyle]
0x001011d4:5b:	RSP(0x001011d4:5b) = RSP(i) + #0xffffffffffffffe0
0x001011d4:5c:	*(ram,RSP(0x001011d4:5b)) = #0x1011d9
0x001011d4:6b:	u0x1000001a:1(0x001011d4:6b) = *(ram,RSP(0x001011d4:5b))
0x001011d4:5d:	call jputchar(free)(#0x34:4,u0x1000001a:1(0x001011d4:6b))
0x001011d9:5e:	goto Block_6:0x001011e5
        \end{lstlisting}
        \vspace{1mm}
    \end{minipage}
    \\ \hline

    % --- ROW 5 (Block 4) ---
    \begin{minipage}[t]{\linewidth}
        \vspace{1mm}
        \begin{lstlisting}[style=CStyle]
} else {
  putchar(L'3');
}
        \end{lstlisting}
        \vspace{1mm}
    \end{minipage}
    & 
    \begin{minipage}[t]{\linewidth}
        \vspace{1mm}
        \textbf{\scriptsize Basic Block 4}
        \begin{lstlisting}[style=PCodeStyle]
0x001011b6:33:	RSP(0x001011b6:33) = RSP(i) + #0xffffffffffffffe0
0x001011b6:34:	*(ram,RSP(0x001011b6:33)) = #0x1011bb
0x001011b6:6d:	u0x10000023:1(0x001011b6:6d) = *(ram,RSP(0x001011b6:33))
0x001011b6:35:	call jputchar(free)(#0x33:4,u0x10000023:1(0x001011b6:6d))
0x001011bb:36:	goto Block_5:0x001011db
        \end{lstlisting}
        \vspace{1mm}
    \end{minipage}
    \\ \hline

    % --- ROW 6 (Block 5) ---
    \begin{minipage}[t]{\linewidth}
        \vspace{1mm}
        \begin{lstlisting}[style=CStyle]
putchar(L'5');
}
        \end{lstlisting}
        \vspace{1mm}
    \end{minipage}
    & 
    \begin{minipage}[t]{\linewidth}
        \vspace{1mm}
        \textbf{\scriptsize Basic Block 5}
        \begin{lstlisting}[style=PCodeStyle]
0x001011e0:38:	RSP(0x001011e0:38) = RSP(i) + #0xffffffffffffffe0
0x001011e0:39:	*(ram,RSP(0x001011e0:38)) = #0x1011e5
0x001011e0:6f:	u0x1000002c:1(0x001011e0:6f) = *(ram,RSP(0x001011e0:38))
0x001011e0:3a:	call jputchar(free)(#0x35:4,u0x1000002c:1(0x001011e0:6f))
        \end{lstlisting}
        \vspace{1mm}
    \end{minipage}
    \\ \hline

    % --- ROW 7 (Block 6) ---
    \begin{minipage}[t]{\linewidth}
        \vspace{1mm}
        \begin{lstlisting}[style=CStyle]
LAB_001011e5:
putchar(L'6');
return;
}
        \end{lstlisting}
        \vspace{1mm}
    \end{minipage}
    & 
    \begin{minipage}[t]{\linewidth}
        \vspace{1mm}
        \textbf{\scriptsize Basic Block 6}
        \begin{lstlisting}[style=PCodeStyle]
0x001011ea:3c:	RSP(0x001011ea:3c) = RSP(i) + #0xffffffffffffffe0
0x001011ea:3d:	*(ram,RSP(0x001011ea:3c)) = #0x1011ef
0x001011ea:71:	u0x10000035:1(0x001011ea:71) = *(ram,RSP(0x001011ea:3c))
0x001011ea:3e:	call jputchar(free)(#0x36:4,u0x10000035:1(0x001011ea:71))
0x001011f1:44:	return(#0x0)
        \end{lstlisting}
        \vspace{1mm}
    \end{minipage}
    \\ \hline

\end{longtable}

Basicblocks are created in \texttt{flow.cc} by the function \texttt{FlowInfo::splitBasic()}. 
The routine partitions the P-code instruction stream at control-flow boundaries: conditional and unconditional jumps, call sites that alter control flow, and return instructions.
Each such instruction ends the current block and/or starts a new one (targets of jumps also begin blocks).

\subsection{The Structuring Algorithm}
\label{sec:stralg}
To transform the \ac{CFG} into C statements, Ghidra employs a structuring algorithm implemented in the 
\texttt{ActionBlockStructure} class. The process involves identifying regions of the graph that match known schemas (or patterns) of control flow:

inside the apply method of \texttt{ActionBlockStructure} we have a call to collapseAll() that is the main loop of the algorithm:

\begin{lstlisting}
void CollapseStructure::collapseAll(void)
{
  int4 isolated_count;

  finaltrace = false;
  graph.clearVisitCount();
  orderLoopBodies();

  collapseConditions();

  isolated_count = collapseInternal((FlowBlock *)0);
  while(isolated_count < graph.getSize()) {
    FlowBlock *targetbl = selectGoto();
    isolated_count = collapseInternal(targetbl);
  }
}
\end{lstlisting}

The method implements a deterministic sequence of passes that progressively transform 
the BasicBlocks into structured FlowBlocks and performs the following steps:

\begin{enumerate}
    \item \textbf{Preparation}\\
        The algorithm first clears previous visitation state (\texttt{graph.clearVisitCount()}) 
        and invokes \texttt{orderLoopBodies()}. This pass discovers loop headers and 
        back-edges, establishing a partial ordering among loop bodies. 
        Detecting loops early is essential to prevent later structuring passes 
        from erroneously breaking loop semantics.

    \item \textbf{Conditional simplification}\\
        Next, \texttt{collapseConditions()} attempts to simplify complex boolean logic 
        and fold adjacent blocks that form logical AND/OR patterns (for example, 
        transforming sequences that represent \texttt{if (A \&\& B)} or \texttt{if (A || B)} 
        into single conditional constructs). This phase applies local rules such as \texttt{ruleBlockOr} 
        to reduce predicate complexity before higher-level structuring.

    \item \textbf{Initial collapse}\\
        The engine then calls \texttt{collapseInternal((FlowBlock *)0)}, which scans the graph 
        and applies standard structuring rules (e.g. \texttt{ruleBlockIfElse}, \texttt{ruleBlockWhileDo}, 
        \texttt{ruleBlockSwitch}) to collapse perfectly structured regions. 
        The routine returns an \texttt{isolated\_count} indicating how many blocks have been 
        fully resolved without introducing gotos.

    \item \textbf{Unstructured flow handling}\\
        If the graph is not fully collapsed (\texttt{isolated\_count < graph.getSize()}), 
        the method iterates: it selects a problematic edge with \texttt{selectGoto()} and 
        marks that edge as unstructured (to be emitted as a \texttt{goto}/\texttt{break}/\texttt{continue} 
        in the final code). The selection is driven by heuristics  
        to minimize disruption to surrounding structure. After marking the edge, 
        \texttt{collapseInternal(targetbl)} is invoked again (often passing the target block of the newly created goto) 
        so the structuring engine can resume collapsing other regions. This loop repeats until every block 
        is resolved.
\end{enumerate}

in the \texttt{collapseInetrnal()} method we have the main pattern recognition method, 
some patterns have precedence over others, since it may occur that a region matches 
multiple schemas. For example, a \texttt{switch} may also match an \texttt{if-else} pattern. 

These are the preferred patterns, in order:
\begin{itemize}
    \item \texttt{goto}
    \item \texttt{cat} (block concatenation)
    \item \texttt{proper if} (if without else)
    \item \texttt{if-else}
    \item \texttt{while-do}
    \item \texttt{do-while}
    \item \texttt{infinite loop}
    \item \texttt{switch}
\end{itemize}
This "rules" are implemented inside a loop that tryes every pattern till no more changes are possible.

in the \texttt{ruleBlockWhileDo()} method we can see how the pattern matching is done:

\begin{lstlisting}

bool CollapseStructure::ruleBlockWhileDo(FlowBlock *bl)

{
  FlowBlock *clauseblock;
  int4 i;

  if (bl->sizeOut() != 2) return false; // Must be binary condition
  if (bl->isSwitchOut()) return false;
  if (bl->getOut(0) == bl) return false; // No loops at this point
  if (bl->getOut(1) == bl) return false;
  if (bl->isInteriorGotoTarget()) return false;
  if (bl->isGotoOut(0)) return false;
  if (bl->isGotoOut(1)) return false;
  for(i=0;i<2;++i) {
    clauseblock = bl->getOut(i);
    if (clauseblock->sizeIn() != 1) continue; // Nothing else must hit clause
    if (clauseblock->sizeOut() != 1) continue; // Only one way out of clause
    if (clauseblock->isSwitchOut()) continue;
    if (clauseblock->getOut(0) != bl) continue; // Clause must loop back to bl

    bool overflow = bl->isComplex(); // Check if we need to use overflow syntax
    if ((i==0)!=overflow) {			// clause must be true out of bl unless we use overflow syntax
      if (bl->negateCondition(true))
	dataflow_changecount += 1;
    }
    BlockWhileDo *newbl = graph.newBlockWhileDo(bl,clauseblock);
    if (overflow)
      newbl->setOverflowSyntax();
    return true;
  }
  return false;
}
\end{lstlisting}

Firstly is checked that the block has exactly two outgoing edges (a binary condition) and is not already part of a switch or a loop. 
Then, for each outgoing edge, it checks if the clauseblock (the potential loop body) has exactly one incoming edge 
(from the condition block) and one outgoing edge (back to the condition block). 
If these conditions are met, it confirms the presence of a while-do loop structure.

A condition is considered complex when the basic block that computes it contains too many instructions to be cleanly represented within a single conditional expression.
The method \texttt{BlockBasic::isComplex()} performs this check.

Criteria: the algorithm counts the number of \textbf{statements} in the block:
\begin{itemize}
    \item A conditional jump (branch) counts as 1 statement.
    \item CALL instructions count as 1.
    \item Operations that produce outputs used only inside the block or marker instructions do not count, 
    but if a variable is used many times or is tied to memory, it contributes to the count.
    \item Threshold: if the total number of statements in the block exceeds 2, the block is considered complex.
\end{itemize}

The overflow syntax (or \texttt{f\_whiledo\_overflow}) is a specific state assigned to a \texttt{BlockWhileDo} when its loop control condition is determined to be complex.
It indicates that, although a logical \texttt{while} structure exists, the conditional block is too long or complicated to be emitted as a single boolean expression \texttt{while(condition)}.
Instead of printing \texttt{while(<complex condition>)\{...\}}, the decompiler emits an alternative form, typically an infinite loop with an internal \texttt{break} to preserve semantics.

\subsection{The $for$ special case}
As can be seen in section \ref{sec:stralg}, the Ghidra decompiler does not have an explicit rule to recognize \texttt{for} loops. 
Indeed, \texttt{for} loops in Ghidra are treated as special cases of \texttt{while-do} loops\footnote{The transformation is triggered only if the architecture option \texttt{analyze\_for\_loops} is enabled.}:
The check is performed in the method \texttt{BlockWhileDo::finalTransform}, this method proceeds only if the block is not marked with overflow syntax.
\begin{enumerate}
    \item \textbf{Loop variable identification:} \texttt{findLoopVariable} is called to search for a variable controlling the iteration (e.g., \texttt{i} in \texttt{i < 10}). 
    This variable must appear in the exit condition and be modified within the loop body.
    \item \textbf{Initializer identification:} \texttt{findInitializer} searches for the instruction that initializes the variable (e.g., \texttt{i = 0}) 
    in the block immediately preceding the loop.
    \item \textbf{Opcode relocation:} If both an iterator (\texttt{iterateOp}) and an initializer (\texttt{initializeOp}) are found, 
    the decompiler physically moves the P-code operations (using \texttt{opUninsert} / \texttt{opInsertAfter}) so they lie adjacent to the loop boundaries, 
    preparing them for syntactic emission.
    \item \textbf{Non-printing marking:} In \texttt{finalizePrinting} these operations are marked with \texttt{opMarkNonPrinting}. 
    This instructs the emitter not to print them as separate statements inside the body or before the loop, but to include them in the \texttt{for(...)} header.
\end{enumerate}

\begin{lstlisting}
{
  // Simplification style
  BlockGraph::finalTransform(data);
  if (!data.getArch()->analyze_for_loops) return;
  if (hasOverflowSyntax()) 
      return; // Still too complex
  FlowBlock *copyBl = getFrontLeaf();
  if (copyBl == (FlowBlock *)0) return;
  BlockBasic *head = (BlockBasic *)copyBl->subBlock(0);
  if (head->getType() != t_basic) return;
  PcodeOp *lastOp = getBlock(1)->lastOp();	// There must be a last op in body, for there to be an iterator statement
  if (lastOp == (PcodeOp *)0) return;
  BlockBasic *tail = lastOp->getParent();
  if (tail->sizeOut() != 1) return;
  if (tail->getOut(0) != head) return;
  PcodeOp *cbranch = getBlock(0)->lastOp();
  if (cbranch == (PcodeOp *)0 || cbranch->code() != CPUI_CBRANCH) return;
  if (lastOp->isBranch()) {			// Convert lastOp to -point- iterateOp must appear after
    lastOp = lastOp->previousOp();
    if (lastOp == (PcodeOp *)0) return;
  }

  findLoopVariable(cbranch, head, tail, lastOp);
  if (iterateOp == (PcodeOp *)0) return;

  if (iterateOp != lastOp) {
    data.opUninsert(iterateOp);
    data.opInsertAfter(iterateOp, lastOp);
  }

  // Try to set up initializer statement
  lastOp = findInitializer(head, tail->getOutRevIndex(0));
  if (lastOp == (PcodeOp *)0) return;
  if (!initializeOp->isMoveable(lastOp)) {
    initializeOp = (PcodeOp *)0;		// Turn it off
    return;
  }
  if (initializeOp != lastOp) {
    data.opUninsert(initializeOp);
    data.opInsertAfter(initializeOp, lastOp);
  }
}
\end{lstlisting}

If all conditions are met, the decompiler effectively transforms the \texttt{while-do} structure into a \texttt{for} loop by relocating and marking the relevant P-code operations.

\subsection{The $Goto$ Problem}
A significant limitation of this approach arises when the \ac{CFG} contains irreducible control flow 
that does not match any predefined schema. (This is common in binaries optimized with aggressive compiler 
techniques or those containing manual assembly optimizations).

When \texttt{ActionBlockStructure} fails to find a matching pattern, the jump inside the FlowBlock remains and it 
will be represented as a \textbf{goto} \footnote{Or a \texttt{break}/\texttt{continue} if it jumps out of/into a loop structure}.
statement to preserve semantic correctness, this phenomenon significantly degrades the readability of the output.

\section{Code Emission}
\label{sec:c_emission}

The final phase of the pipeline is the translation of the structured High P-code into C syntax. 
This is not a simple text dump but a structured generation of an Abstract Syntax Tree (AST) represented 
by \texttt{}{ClangToken} objects.

Before emission, the \textbf{ActionNameVars} pass attempts to assign meaningful names to the 
recovered \texttt{HighVariable} objects. If debug symbols (DWARF, PDB) are available, 
they are utilized. In their absence, Ghidra relies on heuristics based on variable usage 
(e.g., loop counters named \texttt{i}, \texttt{j}) or storage location 
(e.g., \texttt{iVar1}, \texttt{uVar2}). This process is highly stochastic and often results in generic, 
non-descriptive identifiers.

The C++ backend generates a stream of \texttt{ClangToken} objects representing the code structure. This tokenized representation is 
sent to the Java frontend via the XML protocol. This structured data allows the Ghidra 
\ac{GUI} to provide interactive features—such as cross-referencing and dynamic renaming—since the UI 
elements remain linked to the underlying \texttt{Varnode} and \texttt{HighVariable} objects.

\section{LLM}

\subsection{perplexity}

\chapter{Methodology}\label{ch:method}

The framework of our work is based on a client-server architecture, where the server hosts the \ac{LLM} and provides an \ac{API} for interacting with it, 
while the client is responsible for building specific Ghidra version, preparing the code samples and prompts, invoking the server, and collecting results in \ac{JSON} format.
Every service is designed to be modular, allowing for the integration of different \ac{LLM} models and evaluation metrics, for
reproducibility we used \emph{Docker Compose} to containerize both the server and client components, ensuring consistent environments across different machines and operating systems.
The dataset creation is also a containerized process, and the result are mounted as volumes to the client container, allowing for easy access and manipulation of the data without the need for complex data transfer mechanisms.

\section{Dataset Maker}
As written on related works~\ref{sec:llmbench}, we decided to use a subsect of the complete OSS Fuzz dataset used by DecompileBench~\cite{Paper:gao2025decompilebenchcomprehensivebenchmarkevaluating} for our evaluation,
specifically the four Open Source projects: \texttt{file, libxls, readstat, xz}, which are written in C and have a rich history of commits and pull requests on GitHub.
These choice was motivated by the need to have a manageable dataset size for local evaluation, while still covering a set of real world code and functions to evaluate our approach.
for every project the dataset maker extracts all the functions and recompiles them  into standalone binaries; this process create different optimization levels of the binary, specifically \texttt{-O0} and \texttt{-O2}, and \texttt{-O3} 
which are the most common optimization levels used in real world scenarios, and which can have a significant impact on the decompilation output and its readability.

For obtaining this we had to fork the original dataset maker script and modify it to fit our edits; such as the specifically optimization levels (in the original repo they were using all the optimization levels) and some bug fix as pointed out by one pull request on the original repo~\cite{site:githubVarietyProblems}.
so we clone our fork into a container (wich will also run docker inside for building the projects), edited with the patches as shown in the \texttt{README} file of DecompileBench, selected just our four projects and then run the dataset maker script.

\subsection{Dataset Collection}
The result of the dataset maker is a folder named \texttt{Dataset} wich contains other three subfolders:
\begin{itemize}
    \item \textbf{binary}: contains the compiled binary of the functions, every file is named with the format \texttt{task\_project\_functionName-OX.so}, and can be used for decompilation and evaluation.
    \item \textbf{compiled\_ds}: contains a file structure of the dataset format used by the \texttt{Datasets} library~\cite{site:pypiDatasets}, which is a Python library for handling large datasets in a efficient way, and which we use for loading the dataset in our client code. 
        This structure have a file `.arrow' that store data and two \ac{JSON} files for the metadata such as field names and types. In our case we are interested only in three fields \texttt{file}, which contain the name of the function, \texttt{path} wich contains the path `binary/namefile', and \texttt{func} wich contains the source code of the function.
    \item \textbf{eval}: contains also a dataset structure, but we will not use it since is used for recompile success and other metrics that we are not interested in, since we want to focus on the evaluation of the decompilation output rather than the compilation process.
\end{itemize}

\section{LLM Server}

The server is responsible for hosting the \ac{LLM} and providing an \ac{API} for interacting with it, specifically for receiving code samples and prompts from the client, processing them with the \ac{LLM}, and returning the results.
The server is designed to be modular, allowing for the integration of different \ac{LLM} models and evaluation metrics, and it is containerized using Docker for reproducibility and ease of deployment.

It uses Gnunicorn as the WSGI HTTP server for handling incoming requests, and it is built on top of a Python web framework (Flask) to define the \ac{API} endpoints and handle the logic for processing requests and interacting with the \ac{LLM}.
The server is configured with a single worker and thread to manage sequential requests with an extended timeout of 800 seconds to accommodate longer inference times (but more importantly the model loading and the context switch when changing models).
On startup, it performs \ac{GPU} availability checks and pre-downloads all required model weights to the container's cache directory using the Hugging Face libraries.

\subsection{Models}
The heavy part of the framework is without doubt the server, and the models that runs on it.
In our case the local enviroment is a single \ac{GPU} machine with 16 GB of \ac{VRAM}, so we had to select models that can run on this hardware, and that can provide a good performance for our evaluation.
We also used the \myhref{https://apxml.com/tools/VRAM-calculator}{\ac{VRAM} Calculator} to estimate the memory requirements of different models and ensure they fit within our hardware constraints, inside the \ac{VRAM} have to cohesist different areas, such as:
\begin{itemize}
    \item \textbf{Base Model Weights}: The trained parameters of the model, the `weights' with their precision (could be quantizated for reduced memory usage).
    \item \textbf{Activations}: Intermediate computation results during forward passes through the layers. This grows with batch size and input length, and is critical for stability during inference.
    \item \textbf{KV Cache}: Key-Value cache used to avoid recomputing attention for previously processed tokens. Given the lengthy decompilation prompts containing source code, this cache grows proportionally with input length.
    \item \textbf{Framework Overhead}: Fixed memory cost from PyTorch, CUDA drivers, and buffer management. This overhead exists regardless of model size.
\end{itemize}

Based on these considerations, we selected the following models for our evaluation:
\begin{itemize}

\item \textbf{Meta Llama 3.1 (8B)}: 

\item \textbf{Qwen2.5-Coder-Instruct (7B)}: 

\item \textbf{DeepSeek-R1-Distill-Qwen (7B)}: 

\item \textbf{Google Gemma 2 (9B)}: 

...
\end{itemize}

These models were chosen for their balance between performance and memory requirements and for their close number of parameters, allowing us to run them on our local hardware while still providing meaningful insights into the evaluation of decompilation output.

\subsection{Configuration}
The server supports multiple local \acp{LLM} through a simple configuration layer that maps a short, client-facing identifier to the corresponding Hugging Face repository ID. Concretely, a dictionary (\texttt{MODELS\_CONFIG}) defines the available models (e.g., \texttt{qwen-coder}, \texttt{deepseek-r1}, \texttt{llama3.1}, \texttt{gemma2}) and is the single source of truth for both the \texttt{/models} endpoint and for request-time model switching.

For ensure lightness, all models are loaded using 4-bit quantization via \texttt{bitsandbytes}.

\begin{verbatim}
bnb_config = BitsAndBytesConfig(
  load_in_4bit=True,
  bnb_4bit_quant_type="nf4",
  bnb_4bit_use_double_quant=True,
  bnb_4bit_compute_dtype=torch.bfloat16
)
\end{verbatim}

To reduce cold-start delays, the container optionally pre-downloads all model snapshots at startup using \texttt{snapshot\_download}, ensuring that evaluation runs are not affected by network variability. 
Finally, since only one model can reside in GPU memory at a time, the server unloads the currently active model (explicit \texttt{del} + garbage collection + CUDA cache cleanup) 
before loading a new one.

\subsection{Decoding strategy (temperature and top-p)}
For the \texttt{/generate} endpoint, we configured the decoding parameters through a dedicated function that returns a dictionary of \texttt{transformers} generation arguments. 
In our experiments we rely on \emph{nucleus sampling} (\texttt{top\_p}) with a low \texttt{temperature}, to balance determinism (useful for fair comparisons across decompilers) 
and the ability to escape repetitive or low-quality completions.

Concretely, the default configuration is:
\begin{itemize}
    \item \texttt{temperature=0.4}: reduces randomness by sharpening the token distribution. Lower values make outputs more stable across runs, which is desirable for evaluation.
    \item \texttt{top\_p=0.9}: nucleus sampling, i.e., tokens are sampled only from the smallest set whose cumulative probability mass is $p$. This prevents the model from selecting very unlikely tokens while still allowing variation.
    \item \texttt{max\_new\_tokens=2048}: upper bound on completion length, used as a safety and latency-control measure.
\end{itemize}

\subsection{Routes}
The server exposes a minimal \ac{REST} \ac{API}, All endpoints exchange \ac{JSON} payloads and are intentionally kept coarse-grained (a lot of work in a single request) to decouple the client implementation from model-specific details. The available routes are:

\begin{itemize}
    \item \textbf{GET /}: health check endpoint. It returns the server readiness status, whether CUDA is available, and the currently loaded model identifier (if any). This is used by docker compose to ensure healthcheck status for required services.
    \item \textbf{GET /models}: returns the list of supported model keys (the abstract identifiers used by the client), mapped server-side to Hugging Face repository IDs.
    \item \textbf{POST /generate}: main inference endpoint. The request body includes \texttt{model\_id} and a \texttt{prompt}. The server loads (or switches to) the requested model, wraps the prompt into a chat-style template via the tokenizer, runs text generation, and returns the generated completion.
    \item \textbf{POST /score}: scoring endpoint used to compute a language-model based score for a given text. The request body includes \texttt{model\_id} and \texttt{text}. The server computes the token-level negative log-likelihood and returns the derived perplexity.
    \item \textbf{POST /free}: explicit cleanup endpoint to unload the currently resident model and aggressively release GPU memory.
\end{itemize}

Since different models cannot fit simultaneously in \ac{GPU} memory, model switching is handled server-side: each request triggers a check on the currently loaded model and, if needed, a full unload/load cycle. To avoid concurrent access to \ac{GPU} state, all inference and scoring operations are protected by a global lock, enforcing sequential execution.

\subsection{Metrics}
To make the evaluation reproducible and to quantify server-side overhead, the server logs per-request performance metrics to a CSV file (\texttt{llm\_metrics.csv}). Each entry includes:

\begin{itemize}
    \item \textbf{Model and operation}: \texttt{model\_id} and \texttt{operation} (\texttt{generate} or \texttt{score}).
    \item \textbf{Latency}: wall-clock duration (seconds) measured around the full operation, including tokenization and \ac{GPU} synchronization.
    \item \textbf{Peak GPU memory}: peak \ac{VRAM} allocated during the operation, obtained via CUDA peak memory statistics.
    \item \textbf{Tokens}: number of prompt/input tokens and number of generated output tokens; these are also used to derive an approximate throughput (tokens per second).
\end{itemize}

Metric collection is implemented via a dedicated monitoring context manager that resets CUDA peak counters before execution and synchronizes the device before reading final statistics. 
This design provides a uniform measurement procedure across both generation and perplexity scoring, and enables later analysis of the impact of model switching, 
prompt length, and decoding configuration on runtime and memory usage.

\section{Client}

The client is responsible for orchestrating the entire evaluation workflow, including building specific Ghidra versions, preparing code samples and prompts, invoking the server for decompilation and scoring, and collecting results in \ac{JSON} format for analysis.
It is designed to be modular and flexible, allowing for easy integration of different evaluation strategies and metrics, and it is containerized using Docker for reproducibility and ease of deployment.

\subsection{Building Ghidra}
The build process is automated via Python scripts that interact with Git and Gradle (we use an ubuntu image for the container).
Firstly we clone and build the Ghidra repository from GitHub, this version is used as the base for all our evaluations.
after building base and extracted the functions from binary, we get the \ac{PR}s number that we want to evaluate against base from a function that calls github \ac{API} and returns the list of all \ac{PR}s of Ghidra.
Then for each \ac{PR} we have to checkout the specific version of Ghidra, for doing this we have a script that takes as input the \ac{PR} number, and then it fetches the specific head reference from the GitHub repository (\texttt{pull/ID/head:pr-ID}) and checks it out.

For building Ghidra are necessary two prerequisites: \texttt{Java 11} and \texttt{Gradle} (optionally), the first one is required for running the build scripts and the second one is used for managing dependencies and building the project, 
but since Ghidra in newer versions includes a wrapper for Gradle (\texttt{gradlew}), you can use it without installing Gradle globally.

One problem is that every version of Ghidra need a specific version of Java, so we have to check the \texttt{application.properties} file inside the repo for the required minimal Java version, and then install it in the container before building Ghidra.
So inside the container we manage more than one version of Java, and we switch between them based on the requirements of the Ghidra version we are building.
Another problem is that some \ac{PR}s are based on older versions of Ghidra wich does not have the gradle wrapper, so for building those versions we have to do the same thing we have done with Java, but for Gradle, 
we have to install more than one version of Gradle and switch between them based on the requirements of the Ghidra version we are building; This only if \texttt{gradlew} is not available since is more preferible running it instead.
This is the main reason for using an Ubuntu image for the container, since it allows us to easily manage multiple versions of Java and Gradle using the package manager and environment variables.

After building a specific version of Ghidra (Base or \ac{PR}), for every binary found in the dataset folder, we check if it is not already decompiled by that specific version of Ghidra (i.e., if the corresponding \ac{JSON} file with the decompilation output does not exist), 
after creating the list of the files not yet decompiled, we start the decompilation process.

\subsection{Ghidra Headless}

\cite{site:githubGhidraGhidraRuntimeScriptsCommonsupportanalyzeHeadlessREADMEmdMaster}


\subsection{Evaluation}

...

\subsection{Abstraction and Anonymization}
To evaluate the structural quality of the decompilation independently of variable naming and formatting, we implemented an abstraction mechanism using \texttt{tree-sitter} a 
Parser used by \texttt{ATOM}~\cite{site:wikipediaAtomtext}, specifically the language for C with \texttt{tree-sitter-c}.
The Python client parses the decompiled C code into an \ac{AST} and traverses it to generate a `skeletal' representation of the code.

In this representation, specific identifiers, literals, and types are replaced with generic placeholders (e.g., \texttt{id}, \texttt{num}, \texttt{type}), while control flow keywords (\texttt{if}, \texttt{while}, \texttt{for}, \texttt{switch}, \texttt{goto}) and block structures are preserved.
This process effectively anonymizes the code and create a filter/standard for identation and formatting, cleaning out the possible noise and forcing the \ac{LLM} to focus purely on the control flow logic and structural complexity (e.g., the presence of `goto' statements vs\. structured loops)
rather than being biased by variable names, comments or formatting.

...

\subsection{Prompting}

%\section{Dogbolt}

\chapter{Results}\label{ch:results}



\section{LLM performance}



\section{Perplexity as a Metric for ``Humanness''}

\begin{figure}
    \centering
    \includegraphics[width=0.95\textwidth]{img/ppl_distribution.png}
    \caption{On the left perplexity values across original source code, base code, and pr code. On the right perplexity values across abstracted representations of the same functions.}\label{fig:ppl_distribution}
\end{figure}

In \Cref{fig:ppl_distribution} we can see the distribution of perplexity values for the original source code, the base code, and the pr code, as well as their abstracted representations.
We can observe that the original source code has generally higher perplexity values compared to the decompiled versions,
which is unexpected since the original source code should be more ``natural'' and predictable than the decompiled output.
This suggests that the decompilation process may introduce certain patterns or structures that are more familiar to the language model, leading to lower perplexity scores, while the original source code may contain more variability and less predictable constructs that result in higher perplexity.

Another observation is that the abstracted representations of the code (right side of the figure) tend to have higher perplexity values compared to their original counterparts (left side of the figure).
This is likely because the abstraction process removes specific identifiers and literals, which can make the code less predictable and more ``surprising'' to the language model, 
but even in this case, the original source code still has higher perplexity than the decompiled versions, reinforcing the idea that the decompilation process may be introducing more predictable patterns into the code.

... study for trying to know why this happens ...



\section{LLM-as-a-Judge Evaluation}

\begin{figure}
    \centering
    \includegraphics[width=0.95\textwidth]{img/winner_distribution.png}
    \caption{Distribution of winners across all evaluated models.}\label{fig:winner_distribution}
\end{figure}

In \Cref{fig:winner_distribution} we can see the distribution of winners across all evaluated models, based on the qualitative judgments of the \ac{LLM} as a judge.
The two models, Qwen3 and DeepSeek, have a significant number of Ties, but they also show a balanced distribution of wins between the base code and the pr code.
We can see a bin for ``Error'' winners as well, this happens primarly when the model exceed the token limit of 4096 tokens (reasonable limit set by us), since they are reasoning models (they create a context with the generate tokens inside $</think>$ tags) sometimes the context becomes too large (often because they start to repeating thoughts) and they finish the limit without giving inside the response a clear winner (e.g., ``Winner'':``X'').

We previously said that a ``Tie'' is when the model judges always the same result regardless the switch of the base and pr code.
We have a significant number of Ties for all models, watching the result we can observe a ratio of $8.7$ ($618/71$) beetween the number of times that the \ac{LLM} prefers the \ac{PR} version regardless the content.
This suggests a strong bias towards the ``newer'' code, even without telling in the prompt that the Diff code is the newer one, this bias could be due to the fact that Diff code is often more recent and may contain improvements or bug fixes that make it more appealing to the model, or it could be a bias in the model itself towards preferring changes.
Unfortunatly for highlight changes through versions and save on the context window the Diff method is the most convenient, forcing us to Do not count tie-break results in our analysis, giving up almost half of the results.


\subsection{PR \#8628}

\subsection{PR \#8587}

\subsection{PR \#8161}

\subsection{PR \#7253}

\subsection{PR \#6722}



\section{Vs Human Evaluation}


\section{Discussion}



\chapter{Conclusion}\label{ch:conclusion}

Evaluating the readability and ``human-likeness'' of decompiled code has long been a subjective and labor-intensive challenge in reverse engineering. This thesis explored the viability of using \acp{LLM} to automate this evaluation, investigating both statistical metrics (Perplexity) and qualitative prompting strategies (LLM-as-a-Judge).

Our quantitative analysis definitively demonstrated that perplexity is not a reliable proxy for human-likeness in decompiled code. Contrary to initial assumptions, human-written source code consistently exhibits higher perplexity than decompiled output. This is because human code contains domain-specific idioms, creative structural choices, and stylistic variance (high entropy), whereas decompilers generate rigid, repetitive, and verbose boilerplate that artificially drives perplexity down. Thus, the correlation between $\delta$perplexity and human preference was weak and often inversely related, confirming that perplexity alone cannot capture the nuanced qualities that make code more readable or natural to human developers.

In exploring the LLM-as-a-Judge paradigm, our results highlighted distinct capabilities and vulnerabilities between the evaluated models. \emph{Deepseek-r1} emerged as a highly capable evaluator, achieving an alignment of approximately 74\% with human developers. It consistently recognized and rewarded high-level structural improvements, such as the removal of \texttt{goto} statements or the recovery of natural loops. Conversely, \emph{qwen-3} struggled with severe lexical biases and verbosity, often fixating on variable names or minor type casts. We demonstrated that abstracting the decompiled code into an Abstract Syntax Tree (AST) successfully mitigated these lexical distractions, forcing the models to evaluate purely structural logic and improving their alignment with human reasoning.

Despite these successes, we identified significant limitations. LLMs remain susceptible to analytical hallucinations, occasionally inventing non-existent technical justifications when evaluating logically equivalent snippets or minor stylistic permutations (such as reordering \texttt{switch} cases). Furthermore, the computational cost of processing long contexts for perplexity calculations poses a scalability bottleneck for current hardware.

\section{Limitations}

This study has several limitations that should be acknowledged:
\begin{itemize}
    \item \textbf{Model Size and Architecture:} The LLMs evaluated in this study are relatively small compared to state-of-the-art models. Larger models with more parameters may have different capabilities and biases, which could affect the generalizability of our findings.
    \item \textbf{Dataset Scope:} The evaluation was conducted on a specific set of Ghidra PRs and the Dogbolt dataset. While these datasets were carefully selected to cover a range of scenarios, they may not fully represent the diversity of decompiled code or the variety of transformations applied by different decompilers.
    \item \textbf{Human Subject Experiment:} The human evaluation was conducted with a limited number of participants (11) and a small set of questions (21). A larger and more diverse sample size would provide more robust validation of the LLM judges alignment with human intuition.
    \item \textbf{Subjectivity in Readability:} Code readability is inherently subjective, and different developers may have varying preferences for certain coding styles or patterns. This subjectivity can lead to variability in both human and LLM judgments, making it challenging to establish a definitive ground truth for readability.
    \item \textbf{Perplexity as a Metric:} While we demonstrated that perplexity is not a reliable proxy for human-likeness, it remains a widely used metric in language modeling. Future research should explore alternative metrics that better capture the qualitative aspects of code readability.
\end{itemize}

In conclusion, while LLMs are not yet flawless, objective arbiters of code quality, they represent a powerful and promising tool for the reverse engineering community. The findings of this thesis establish a solid baseline for automated decompiler evaluation. Future work should focus on utilizing larger foundation models, developing specialized fine-tuning datasets composed of human-ranked assembly-to-C translations, and refining AST abstraction pipelines to further isolate structural quality from lexical noise.



% Bibliography using biber; comment this and uncomment the following lines to use bibtex instead
\printbibliography
% \bibliographystyle{alphaurl}
% \bibliography{bib}

\chapter{Appendix}

\section{Quiz questions}

These are the quiz questions that were used in the human evaluation of the decompiled code snippets.

\subsection{Participant instructions}

In this short quiz (about 5--10 minutes), you will be presented with pairs of C code snippets (Snippet A and Snippet B).  
Each pair implements the same logical operation, but with different writing styles.

Your task is to choose which snippet appears more \emph{humanly written}, idiomatic, and maintainable.  
There is no right or wrong answer; we are interested in your developer intuition.  
If you cannot decide, leave the answer blank.

All responses are collected anonymously and used only for academic research.

\subsection{Quiz questions}

\begin{figure}[H]
\centering
\includegraphics[width=\textwidth]{img/quiz/1.png}
\caption{PR \#8161 --- \texttt{task-readstat\_readstat\_convert-O2} (Reason for inclusion: consistency)}
\end{figure}

\begin{figure}[H]
\centering
\includegraphics[width=\textwidth]{img/quiz/2.png}
\caption{PR \#8161 --- \texttt{task-readstat\_sav\_parse\_very\_long\_string\_record-O3} (Reason for inclusion: consistency between DeepSeek; Qwen inconsistency)}
\end{figure}

\begin{figure}[H]
\centering
\includegraphics[width=\textwidth]{img/quiz/3.png}
\caption{PR \#8161 --- \texttt{task-file\_file\_zmagic-O0} (Reason for inclusion: older PR but better output, possibly due to conciseness)}
\end{figure}

\begin{figure}[H]
\centering
\includegraphics[width=\textwidth]{img/quiz/4.png}
\caption{PR \#8628 --- \texttt{task-readstat\_sav\_parse\_date-O3} (Reason for inclusion: inconsistency between \texttt{()} and \texttt{+-})}
\end{figure}

\begin{figure}[H]
\centering
\includegraphics[width=\textwidth]{img/quiz/5.png}
\caption{PR \#8628 --- \texttt{task-readstat\_sav\_parse\_very\_long\_string\_record-O0} (Reason for inclusion: consistency; PR wins)}
\end{figure}

\begin{figure}[H]
\centering
\includegraphics[width=\textwidth]{img/quiz/6.png}
\caption{PR \#8628 --- \texttt{task-readstat\_sav\_parse\_time-O2} (Reason for inclusion: consistency; PR wins)}
\end{figure}

\begin{figure}[H]
\centering
\includegraphics[width=\textwidth]{img/quiz/7.png}
\caption{PR \#6722 --- \texttt{task-readstat\_sas\_rle\_decompress-O3} (Reason for inclusion: PR consistency and tie outcome)}
\end{figure}

\begin{figure}[H]
\centering
\includegraphics[width=\textwidth]{img/quiz/8.png}
\caption{PR \#6722 --- \texttt{task-file\_file\_zmagic-O3} (Reason for inclusion: unusual \texttt{if} structure; PR wins)}
\end{figure}

\begin{figure}[H]
\centering
\includegraphics[width=\textwidth]{img/quiz/9.png}
\caption{PR \#6722 --- \texttt{task-file\_file\_signextend-O0} (Reason for inclusion: inconsistency)}
\end{figure}

\begin{figure}[H]
\centering
\includegraphics[width=\textwidth]{img/quiz/10.png}
\caption{PR \#7253 --- \texttt{task-readstat\_readstat\_parse\_por-O3} (Reason for inclusion: difficult to assess)}
\end{figure}

\begin{figure}[H]
\centering
\includegraphics[width=\textwidth]{img/quiz/11.png}
\caption{PR \#8587 --- \texttt{task-readstat\_dta\_parse\_timestamp-O2} (Reason for inclusion: PR consistency)}
\end{figure}

\begin{figure}[H]
\centering
\includegraphics[width=\textwidth]{img/quiz/12.png}
\caption{PR \#8587 --- \texttt{task-readstat\_sav\_parse\_date-O2} (Reason for inclusion: the only inconsistency in the PR)}
\end{figure}

\begin{figure}[H]
\centering
\includegraphics[width=\textwidth]{img/quiz/13.png}
\caption{PR \#7252 --- \texttt{task-xz\_lzma\_validate\_chain-O2} (Reason for inclusion: DeepSeek inconsistency and Qwen error)}
\end{figure}

\begin{figure}[H]
\centering
\includegraphics[width=\textwidth]{img/quiz/14.png}
\caption{PR \#8587 --- \texttt{task-file\_file\_replace-O0} (Reason for inclusion: non-standard modification; PR wins)}
\end{figure}

\begin{figure}[H]
\centering
\includegraphics[width=\textwidth]{img/quiz/15.png}
\caption{PR \#7252 --- \texttt{task-libxls\_xls\_parseWorkBook-O0} (Reason for inclusion: single-cast modification, but with inconsistency)}
\end{figure}

\begin{figure}[H]
\centering
\includegraphics[width=\textwidth]{img/quiz/16.png}
\caption{Dogbolt --- \texttt{task-file\_file\_magicfind-O2} (Reason for inclusion: Ghidra vs Hex-Rays comparison)}
\end{figure}

\begin{figure}[H]
\centering
\includegraphics[width=\textwidth]{img/quiz/17.png}
\caption{Dogbolt --- \texttt{task-file\_file\_is\_simh-O0} (Reason for inclusion: Binary Ninja vs Hex-Rays comparison)}
\end{figure}

\begin{figure}[H]
\centering
\includegraphics[width=\textwidth]{img/quiz/18.png}
\caption{Dogbolt --- \texttt{task-file\_file\_is\_simh-O3} (Reason for inclusion: Binary Ninja vs Ghidra comparison)}
\end{figure}

\begin{figure}[H]
\centering
\includegraphics[width=\textwidth]{img/quiz/19.png}
\caption{Code --- \texttt{task-file\_file\_default-O2} (Reason for inclusion: Ghidra vs Hex-Rays comparison)}
\end{figure}

\begin{figure}[H]
\centering
\includegraphics[width=\textwidth]{img/quiz/20.png}
\caption{Code --- \texttt{task-file\_magic\_buffer-O0} (Reason for inclusion: Binary Ninja vs Hex-Rays comparison)}
\end{figure}

\begin{figure}[H]
\centering
\includegraphics[width=\textwidth]{img/quiz/21.png}
\caption{Code --- \texttt{task-file\_fmtcheck-O0} (Reason for inclusion: Binary Ninja vs Ghidra comparison)}
\end{figure}

\section{Future Ghidra work}

A possible future direction, based on the ``groups'' mechanism described in the \Cref{sec:dgroups}, is to define a standard way to enable alternative action pipelines in the decompiler.
In practice, existing actions could include guarded branches (e.g., \texttt{if}-based checks) to activate different transformation flows and generate multiple candidate C outputs for the same function.

\begin{lstlisting}

void BlockWhileDo::finalTransform(Funcdata &data){
  // Simplification style
  const string &currentAction = data.getArch()->allacts.getCurrentName();
  BlockGraph::finalTransform(data);
  if (!data.getArch()->analyze_for_loops) return;
  if (hasOverflowSyntax()) 
    if (currentAction == "different"){// LukeSerne example of PR-specific transformation
      // Apply a different set of transformations to generate an alternative decompilation variant
    [...]
 }else
      return; // Still too complex
  FlowBlock *copyBl = getFrontLeaf();
    [...] // Original transformation flow for the standard decompilation variant
}
\end{lstlisting}


On top of this, a dedicated Ghidra extension could orchestrate the process by:
\begin{itemize}
    \item collecting all generated decompilation variants,
    \item sending them to an LLM through an API call for ranking/selection,
    \item and showing to the analyst only the variant selected as the most readable or human-like.
\end{itemize}

We have prototyped this architecture in a separate project, and the following \Cref{ext_code,ext_ui} show a possible code structure for the extension and a mockup of the user interface.

\begin{figure}[H]
\centering
\includegraphics[width=\textwidth]{img/ext_code.png}
\caption{A possible code architecture for a Ghidra extension that manages multiple decompilation variants and uses an LLM to select the best one.}\label{fig:ext_code}
\end{figure}

\begin{figure}[H]
\centering
\includegraphics[width=\textwidth]{img/ext_ui.png}
\caption{A possible user interface for the Ghidra extension, showing the selected decompilation variant.}\label{fig:ext_ui}
\end{figure}

This would provide a modular framework for experimenting with decompilation strategies while keeping the final user interface simple.

\end{document}

